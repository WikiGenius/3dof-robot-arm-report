%================== sections/04_dynamics.tex ==================
\section{Dynamics of the 3-DOF Arm}\label{sec:dynamics}
% \owner{Muhammed Elyamani} \todo{Derive $M(\vect{q})$, $C(\vect{q},\dot{\vect{q}})$, and $g(\vect{q})$; verify symmetry and positive definiteness of $M(\vect{q})$.}

The dynamic model of the robotic arm can be derived using the Euler--Lagrange formulation. 
Let $\vect{q} \in \mathbb{R}^3$ denote the joint coordinates and $\vect{\tau} \in \mathbb{R}^3$ the actuator torques. 
The general form of the manipulator dynamics is
\begin{equation}
  M(\vect{q})\,\ddot{\vect{q}} + C(\vect{q},\dot{\vect{q}})\,\dot{\vect{q}} + g(\vect{q}) = \vect{\tau},
  \label{eq:dynamics}
\end{equation}
where:
\begin{itemize}[leftmargin=2em]
  \item $M(\vect{q}) \in \mathbb{R}^{3\times3}$ is the symmetric, positive definite \textit{inertia matrix}.
  \item $C(\vect{q},\dot{\vect{q}})\,\dot{\vect{q}}$ represents Coriolis and centrifugal effects.
  \item $g(\vect{q})$ is the gravitational torque vector.
\end{itemize}

The kinetic and potential energies are expressed as
\begin{align}
  T(\vect{q},\dot{\vect{q}}) &= \tfrac{1}{2}\,\dot{\vect{q}}^{\T} M(\vect{q}) \dot{\vect{q}}, \\[3pt]
  V(\vect{q}) &= \text{potential energy due to gravity.}
\end{align}
Substituting into the Lagrangian $L = T - V$ and applying
$\frac{d}{dt}\!\left(\frac{\partial L}{\partial \dot{q}_i}\right) - \frac{\partial L}{\partial q_i} = \tau_i$
yields \eqref{eq:dynamics}.

\begin{tcolorbox}[title=Dynamic Model Validation, colback=white!98!gray]
Ensure that:
\begin{itemize}[noitemsep, topsep=2pt]
  \item $M(\vect{q})$ is symmetric and positive definite (SPD).
  \item $\dot{M}(\vect{q}) - 2C(\vect{q},\dot{\vect{q}})$ is skew-symmetric.
  \item Energy balance holds: $\dot{T} + \dot{V} = \vect{\tau}^\T \dot{\vect{q}}$.
\end{itemize}
\end{tcolorbox}
